\documentclass[aspectratio=169]{beamer}
\usetheme{default}
\usecolortheme{beaver}
\setbeamertemplate{navigation symbols}{} 
\setbeamersize{text margin left=6mm, text margin right=0.2cm} 

\usepackage[utf8]{inputenc}

\title[]{Example document to recreate with beamer in \LaTeX}

\author{Daniëlle Remmerswaal}
\date[]{January 2022}


%% These additional packages are used within the document:
\usepackage{ragged2e}  % `\justifying` text
\usepackage{booktabs}  % Tables
\usepackage{tabularx}
\usepackage{tikz}      % Diagrams
\usetikzlibrary{calc, shapes, backgrounds}
\usepackage{amsmath, amssymb}
\usepackage{url}       % `\url`s
\usepackage{listings}  % Code listings
\frenchspacing
\begin{document}

\begin{frame}[b]{}
\titlepage    
    \begin{center}
    Markup Languages and Reproducible Programming in Statistics
    \end{center}
\end{frame}
  
  
  \begin{frame}
\frametitle{Outline}
\tableofcontents

\end{frame}




\section{Working with equations}
\begin{frame}{Working with equations}
We define a set of equations as
\begin{equation}
    a = b + c^2,
\end{equation}
\begin{equation}
    a-c^2=b,
\end{equation}
\begin{equation}
    \mbox{left side = right side,}
\end{equation}
\begin{equation}
    \text{left side + something} \geq \text{right side,}
\end{equation}
for all something $>$ 0.
\end{frame}

\subsection{Aligning the same equations}
\begin{frame}{Aligning the same equations}
Aligning the equations by the equal sign gives a much better view into the placements
of the separate equation components.    

\begin{align}
    a &= b + c^2,
    \\a-c^2&=b,
    \\\mbox{left side} &= \text{right side,}
    \\\text{left side + something} &\geq \text{right side,}
\end{align}
\end{frame}

\subsection{Omit equation numbering}
\begin{frame}{Omit equation numbering}
Alternatively, the equation numbering can be omitted.
\begin{align*}
    a &= b + c^2
    \\a-c^2&=b
    \\\mbox{left side} &= \text{right side}
    \\\text{left side + something} &\geq \text{right side}
\end{align*}
\end{frame}

\subsection{Ugly alignment}
\begin{frame}{Ugly alignment}
Some components do not look well, when aligned. Especially equations with different
heights and spacing. For example,
\begin{align}
    E&=mc^2,
    \\m&=\frac{E}{c^2},
    \\E&=\sqrt{\frac{E}{m}}.
\end{align}
Take that into account.
\end{frame}

\section{Discussion}
\begin{frame}{Discussion}
    This is where you’d normally give your audience a recap of your talk, where you could
discuss e.g. the following
\begin{itemize}
     \item Your main findings
    \item The consequences of your main findings
    \item Things to do
    \item Any other business not currently investigated, but related to your talk
\end{itemize}
\end{frame}



 \end{document}
